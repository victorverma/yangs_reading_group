\documentclass{beamer}

\mode<presentation> {
\usetheme{AnnArbor}
}

\usepackage{graphicx}
\graphicspath{{./figures/}}
\usepackage{caption}
\usepackage{subcaption}
\usepackage{hyperref}
\hypersetup{colorlinks=true}
\usepackage{amsmath}
\usepackage{amsthm}
\usepackage[shortlabels]{enumitem}
\usepackage{biblatex}
\addbibresource{bibliography.bib}

\newtheorem{proposition}{Proposition}

\title[Point Processes in Extreme Value Theory]{Point Processes in Extreme Value Theory}

\author{Victor Verma}
\institute[]
{
Prof. Yang Chen's Reading Group \\
Department of Statistics \\
University of Michigan
}
\date[2/9/23]{2/9/23} 

\begin{document}

\begin{frame}
    \titlepage
\end{frame}

\begin{frame}{Today's Reading}
    \begin{itemize}
        \item Chapter 5 of \textit{Modelling Extremal Events} by Embrechts, Kl\"{u}ppelberg, and Mikosch (\cite{embrechts_et_al_1997})
    \end{itemize}
\end{frame}

%\begin{frame}{Outline}
%    \tableofcontents
%\end{frame}

\begin{frame}{An Example Problem}
    % insert figure
\end{frame}

\begin{frame}{What is a Point Process?}
    A \textbf{point process} models a random configuration of points in a space $E$.

    \smallskip

    $E$ is assumed to be a topological space that
    \begin{itemize}
        \item is Hausdorff
        \item is locally compact
        \item has a countable basis
    \end{itemize}
    For simplicity, we can assume that $E$ is a subset of a finite-dimensional Euclidean space $\mathbb{R}^d$.

    \smallskip
    
    Let $\mathcal{E}$ be the Borel $\sigma$-algebra on $E$, i.e., the $\sigma$-algebra generated by the open subsets of $E$.
\end{frame}

\begin{frame}{What is a Point Process?}
    For $x \in E$, the \textbf{Dirac measure} $\epsilon_x : \mathcal{E} \to [0, \infty]$ is defined by
    \[
    \epsilon_x(A) =
        \begin{cases}
             1 & \text{if $x \in A$} \\
             0 & \text{if $x \notin A$}
        \end{cases}
    \]

    \smallskip

    Let $\{x_i\}_{i = 1}^{\infty}$ be a sequence in $E$. Define $m : \mathcal{E} \to [0, \infty]$ by
    \[
    m(A) = \sum_{i = 1}^{\infty} \epsilon_{x_i}(A)
    \]
    and assume that $m$ is \textbf{Radon}, i.e., $m(K) < \infty$ for all compact $K \subseteq E$. $m$ is called a \textbf{point measure}.

    \smallskip

    For $E \subseteq \mathbb{R}^d$, the Radon condition is equivalent to the condition that any bounded $A \subseteq E$ contains finitely many $x_i$'s.
\end{frame}

\begin{frame}{What is a Point Process?}
    Let $M_p(E)$ be the set of all point measures on $E$. Let $\mathcal{M}_p(E)$ be the $\sigma$-algebra generated by sets of the form
    \[
    \{m \in M_p(E) : m(A) \in B\},
    \]
    where $A \in \mathcal{E}$ and $B \in \mathcal{B}([0, \infty])$.

    \smallskip

    \begin{definition}
        A \textbf{point process} on $E$ is a measurable map
        \[
        N : (\Omega, \mathcal{F}, P) \to (M_p(E), \mathcal{M}_p(E)),
        \]
        where $(\Omega, \mathcal{F}, P)$ is a probability space.
    \end{definition}
\end{frame}

\begin{frame}{More about $\mathcal{M}_p(E)$}
    $\mathcal{M}_p(E)$ is the smallest $\sigma$-algebra that makes $m \mapsto m(A)$ measurable for all $A \in \mathcal{E}$. Because of this, $N$ is a point process if and only if $N(A)$ is an extended real-valued random variable for all $A \in \mathcal{E}$. More formally,
    \begin{proposition}
    $N : \Omega \to M_p(E)$ is a point process if and only if for all $A \in \mathcal{E}$, $\omega \mapsto N(\omega, A)$ is a measurable mapping from $(\Omega, \mathcal{A})$ to $([0, \infty], \mathcal{B}[0, \infty])$.
    \end{proposition}
\end{frame}

\begin{frame}{A General Way to Create Point Processes}
    Let $\{X_i\}_{i = 1}^{\infty}$ be a sequence of random elements of $E$. Let
    \[
    N = \sum_{i = 1}^{\infty} \epsilon_{X_i}.
    \]
    If $\sum_{i = 1}^{\infty} \epsilon_{X_i(\omega)}$ is Radon for all $\omega$, then $N$ is a point process.
\end{frame}

\begin{frame}{The Point Process of Exceedances}
    Let $\{X_n\}_{n = 1}^{\infty}$ be a sequence of random variables defined on $(\Omega, \mathcal{F}, P)$ and let $u \in \mathbb{R}$. Let $E = (0, 1]$ and $\mathcal{E} = \mathcal{B}(E)$. For $n \in \mathbb{N}$, the \textbf{point process of exceedances} $N_n$ is defined for all $A \in \mathcal{E}$ by
    \[
    N_n(A) = \sum_{i = 1}^n \epsilon_{i / n}(A)I_{\{X_i > u\}}.
    \]
    We have $N_n(E) = \sum_{i = 1}^n \epsilon_{i / n}(E)I_{\{X_i > u\}} = \sum_{i = 1}^n I_{\{X_i > u\}}$, i.e., how many of $X_1, \ldots, X_n$ exceed $u$.
\end{frame}

\begin{frame}{The Point Process of Exceedances}
    Let $X_{1, n} \ge \ldots \ge X_{n, n}$ be the order statistics of $X_1, \ldots, X_n$. We have
    \begin{align*}
        N_n(E) = 0 &\iff X_{1, n} \le u \\
        N_n(E) < k &\iff X_{k, n} \le u
    \end{align*}
    for any $k$.

    \smallskip

    We can write
    \[
    \tilde{N}_n(\cdot) = \sum_{i = 1}^n \epsilon_{i / n, X_i}(\cdot),
    \]
    which has two-dimensional state space $E = (0, 1] \times (u, \infty)$. Then
    \[
    \tilde{N}_n(A \times (u, \infty)) = N_n(A).
    \]
\end{frame}

\begin{frame}{Distributions and Mean Measures}
    The \textbf{distribution} $P_N$ of a point process $N$ is the probability measure $P \circ N^{-1} = P(N \in \cdot)$ defined on $\mathcal{M}_p(E)$.

    \smallskip

    The \textbf{mean measure} $\mu$ of a point process $N$ satisfies
    \begin{align*}
        \mu(A) &= E[N(A)] \\
        &= \int_{\Omega} (N(\omega))(A)\,dP(\omega) \\
        &= \int_{M_p(E)} m(A)\,d(P \circ N^{-1})(m) \\
        &= \int_{M_p(E)} m(A)\,dP_N(m).
    \end{align*}
\end{frame}

\begin{frame}{Laplace Functionals}
    For a measure $m$, write $\int_E g\,dm$ as $m(g)$. The \textbf{Laplace functional} $\Psi_N$ of a point process $N$ is defined by
    \begin{align*}
        \Psi_N(g) &= E[e^{-N(g)}] \\
        &= \int_{\Omega} e^{-N(\omega)(g)}\,dP(\omega) \\
        &= \int_{M_p(E)} e^{-m(g)}\,dP_N(m)
    \end{align*}
    for nonnegative measurable functions $g$ on $E$.
\end{frame}

\begin{frame}{Examples of Laplace Functionals}
    Example 5.1.8
\end{frame}

\begin{frame}{Poisson Processes/Poisson Random Measures}
    A point process $N$ is a \textbf{Poisson process} or \textbf{Poisson random measure (PRM)} with \textbf{mean measure} $\mu$ if
    \begin{enumerate}[(a)]
    \item for all $A \in \mathcal{E}$, for all $k \ge 0$,
    \[
    P(N(A) = k) =
        \begin{cases}
            e^{-\mu(A)}\frac{(\mu(A))^k}{k!} & \text{if $\mu(A) < \infty$} \\
            0 & \text{if $\mu(A) = \infty$}
        \end{cases}
    \]
    \item For all $m \ge 1$, for all disjoint $A_1, \ldots, A_m \in \mathcal{E}$, $N(A_1), \ldots, N(A_m)$ are independent.
    \end{enumerate}
    We say that $N$ is $\text{PRM}(\mu)$.
\end{frame}

\begin{frame}{The Laplace Functional of a PRM}
    Let $N$ be $\text{PRM}(\mu)$. Then for all nonnegative measurable functions $g$ on $E$,
    \[
    \Psi_N(g) = \exp\left\{-\int_E \left(1 - e^{-g(x)}\right)\,d\mu(x)\right\}
    \]
\end{frame}

\section{References}

\begin{frame}[allowframebreaks]{References}
    \nocite{*}
    \printbibliography
\end{frame}

\end{document}