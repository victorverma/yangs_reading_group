\documentclass{beamer}

\mode<presentation> {
\usetheme{AnnArbor}
}

\usepackage{graphicx}
\graphicspath{{./figures/}}
\usepackage{caption}
\usepackage{subcaption}
\usepackage{hyperref}
\hypersetup{colorlinks=true}
\usepackage{amsmath}
\usepackage{amsthm}
\usepackage[shortlabels]{enumitem}
\usepackage{biblatex}
\addbibresource{bibliography.bib}

\title[Point Processes in Extreme Value Theory]{Point Processes in Extreme Value Theory}

\author{Victor Verma}
\institute[]
{
Prof. Yang Chen's Reading Group \\
Department of Statistics \\
University of Michigan
}
\date[2/9/23]{2/9/23} 

\begin{document}

\begin{frame}
    \titlepage
\end{frame}

\begin{frame}{Today's Reading}
    \begin{itemize}
        \item Chapter 5 of \textit{Modelling Extremal Events} by Embrechts, Kl\"{u}ppelberg, and Mikosch (\cite{embrechts_et_al_1997})
    \end{itemize}
\end{frame}

%\begin{frame}{Outline}
%    \tableofcontents
%\end{frame}

\begin{frame}{An Example Problem}
    % insert figure
\end{frame}

\begin{frame}{What is a Point Process?}
    Let $E$ be a subset of a finite-dimensional Euclidean space. $E$ will be the state space. Let $\mathcal{E}$ be the Borel $\sigma$-algebra on $E$.

    \smallskip

    For $x \in E$, the \textbf{Dirac measure} $\epsilon_x : \mathcal{E} \to [0, \infty]$ is defined by
    \[
    \epsilon_x(A) =
        \begin{cases}
             1 & \text{if $x \in A$} \\
             0 & \text{if $x \notin A$}
        \end{cases}
    \]
\end{frame}

\begin{frame}{What is a Point Process?}
    Let $\{x_i\}_{i = 1}^{\infty}$ be a sequence in $E$. Define $m : \mathcal{E} \to [0, \infty]$ by
    \[
    m(A) = \sum_{i = 1}^{\infty} \epsilon_{x_i}(A)
    \]
    and assume that $m(K) < \infty$ for all compact $K \subseteq E$. $m$ is called a \textbf{point measure}.

    \smallskip

    Let $M_p(E)$ be the set of all point measures on $E$. Let $\mathcal{M}_p(E)$ be a particular $\sigma$-algebra on $M_p(E)$.

    \smallskip

    A \textbf{point process} on $E$ is a measurable map
    \[
    N : [\Omega, \mathcal{F}, P] \to [M_p(E), \mathcal{M}_p(E)].
    \]
\end{frame}

\begin{frame}{An Example Point Process}
    Example 5.1.3: the point process of exceedances
\end{frame}

\begin{frame}{Distributions and Laplace Functionals}
    The \textbf{distribution} $P_N$ of a point process $N$ is the probability measure $P \circ N^{-1} = P(N \in \cdot)$ defined on $\mathcal{M}_p(E)$.

    \smallskip

    The \textbf{Laplace functional} $\Psi_N$ of a point process $N$ is defined by
    \begin{align*}
        \Psi_N(g) &= E[e^{-N(g)}] \\
        &= \int_{\Omega} e^{-N(\omega)(g)}\,dP(\omega) \\
        &= \int_{M_p(E)} e^{-m(g)}\,dP_N(m)
    \end{align*}
    for nonnegative measurable functions $g$ on $E$.
\end{frame}

\begin{frame}{Examples of Laplace Functionals}
    Example 5.1.8
\end{frame}

\begin{frame}{Poisson Processes/Poisson Random Measures}
    A point process $N$ is a \textbf{Poisson process} or \textbf{Poisson random measure (PRM)} with \textbf{mean measure} $\mu$ if
    \begin{enumerate}[(a)]
    \item for all $A \in \mathcal{E}$, for all $k \ge 0$,
    \[
    P(N(A) = k) =
        \begin{cases}
            e^{-\mu(A)}\frac{(\mu(A))^k}{k!} & \text{if $\mu(A) < \infty$} \\
            0 & \text{if $\mu(A) = \infty$}
        \end{cases}
    \]
    \item For all $m \ge 1$, for all disjoint $A_1, \ldots, A_m \in \mathcal{E}$, $N(A_1), \ldots, N(A_m)$ are independent.
    \end{enumerate}
    We say that $N$ is $\text{PRM}(\mu)$.
\end{frame}

\section{References}

\begin{frame}[allowframebreaks]{References}
    \nocite{*}
    \printbibliography
\end{frame}

\end{document} 