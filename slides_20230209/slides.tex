\documentclass{beamer}

\mode<presentation> {
\usetheme{AnnArbor}
}

\usepackage{graphicx}
\graphicspath{{./figures/}}
\usepackage{caption}
\usepackage{subcaption}
\usepackage{hyperref}
\hypersetup{colorlinks=true}
\usepackage{amsmath}
\usepackage{amsthm}
\usepackage{biblatex}
\addbibresource{bibliography.bib}

\title[Point Processes in Extreme Value Theory]{Point Processes in Extreme Value Theory}

\author{Victor Verma}
\institute[]
{
Prof. Yang Chen's Reading Group \\
Department of Statistics \\
University of Michigan
}
\date[2/9/23]{2/9/23} 

\begin{document}

\begin{frame}
    \titlepage
\end{frame}

\begin{frame}{Today's Reading}
    \begin{itemize}
        \item Chapter 5 of \textit{Modelling Extremal Events} by Embrechts, Kl\"{u}ppelberg, and Mikosch (\cite{embrechts_et_al_1997})
    \end{itemize}
\end{frame}

%\begin{frame}{Outline}
%    \tableofcontents
%\end{frame}

\begin{frame}{An Example Problem}
    % insert figure
\end{frame}

\begin{frame}{What is a Point Process?}
    Let $E$ be a subset of a finite-dimensional Euclidean space. $E$ will be the state space. Let $\mathcal{E}$ be the Borel $\sigma$-algebra on $E$.

    \smallskip

    For $x \in E$, the \textbf{Dirac measure} $\epsilon_x : \mathcal{E} \to [0, \infty]$ is defined by
    \[
    \epsilon_x(A) =
        \begin{cases}
             1 & \text{if $x \in A$} \\
             0 & \text{if $x \notin A$}
        \end{cases}
    \]
\end{frame}

\begin{frame}{What is a Point Process?}
    Let $\{x_i\}_{i = 1}^{\infty}$ be a sequence in $E$. Define $m : \mathcal{E} \to [0, \infty]$ by
    \[
    m(A) = \sum_{i = 1}^{\infty} \epsilon_{x_i}(A)
    \]
    and assume that $m(K) < \infty$ for all compact $K \subseteq E$. $m$ is called a \textbf{point measure}.

    \smallskip

    Let $M_p(E)$ be the set of all point measures on $E$. Let $\mathcal{M}_p(E)$ be a particular $\sigma$-algebra on $M_p(E)$.

    \smallskip

    A \textbf{point process} on $E$ is a measurable map
    \[
    N : [\Omega, \mathcal{F}, P] \to [M_p(E), \mathcal{M}_p(E)].
    \]
\end{frame}

\section{References}

\begin{frame}[allowframebreaks]{References}
    \nocite{*}
    \printbibliography
\end{frame}

\end{document} 