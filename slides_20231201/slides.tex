\documentclass{beamer}

\mode<presentation> {
\usetheme{AnnArbor}
}

\usepackage{graphicx}
\graphicspath{{./figures/}}
\usepackage{caption}
\usepackage{subcaption}
\usepackage{hyperref}
\hypersetup{colorlinks=true}
\usepackage{amsmath}
\usepackage{amsthm}
\usepackage{biblatex}
\addbibresource{bibliography.bib}

\setbeamertemplate{caption}[numbered]

\newtheorem{proposition}{Proposition}
\def\P{\mathbb P}
\def\ind{\perp\!\!\!\perp}
\DeclareMathOperator*{\argmax}{arg\,max}

\title[Temporal Point Processes]{Temporal Point Processes}

\author{Aniket Jivani \inst{1} and Victor Verma \inst{2}}
\institute[U-M]{
\inst{1} Department of Mechanical Engineering, University of Michigan \and %
\inst{2} Department of Statistics, University of Michigan
}
\date[12/1/23]{12/1/23}

\begin{document}

\begin{frame}
    \titlepage
\end{frame}

%\begin{frame}{Outline}
%    \tableofcontents
%\end{frame}
\section{Counting Processes}
\begin{frame}{Definition}
    
\end{frame}

% add more frames, redo sections etc. etc.


\section{Temporal Point Processes}
\begin{frame}{Poisson Processes}
    
\end{frame}

\begin{frame}{Hawkes Processes}
    
\end{frame}

\begin{frame}{Simulation Methods}
    
\end{frame}

\section{Applications}
\begin{frame}{Solar Flares}
    
\end{frame}

\begin{frame}{Other Examples}
    
\end{frame}

\section{Neural STPP}
\begin{frame}{STPP}
    \emph{Reference:} RTQ Chen et al. 2020\cite{chen_neural_2020}

    \emph{Key Idea:} Use Continuous Normalizing Flows (CNFs) to Compute likelihoods for highly complex spatio-temporal distributions

    \textcolor{red}{Provide better motivation behind use of NNs}
\end{frame}

\begin{frame}{Neural ODEs}
    Neural ODE: Continuous update for hidden states of neural networks by passing them through classical ODE solvers.

    \begin{itemize}
        \item Model arbitrary irregularly sampled time series

        \item Learn dynamics from data for single or multiple simulations (\emph{parametrized} NODE that enables UQ)

        \item Computationally tractable and capable of learning well-regularized flows based on lifting spaces and ideas from optimal transport
    \end{itemize}
\end{frame}

\begin{frame}{Example}
    
\end{frame}

\begin{frame}{Continuous Normalizing Flows}
    Change of variables

    Computational difficulty (show highly constrained architectures)

    Speed up through CNF
\end{frame}

\begin{frame}{Example}
    Gaussian to multi-modal distribution
\end{frame}

\begin{frame}{Proposed framework}
    Need 3-4 slides to set this up
\end{frame}

\begin{frame}{Results}
    Need comparison with a regular estimation procedure
\end{frame}

\begin{frame}{References}
    \nocite{*}
    \printbibliography
\end{frame}

\end{document}