\documentclass{beamer}

\mode<presentation> {
\usetheme{AnnArbor}
}

\usepackage{graphicx}
\graphicspath{{./figures/}}
\usepackage{caption}
\usepackage{subcaption}
\usepackage{hyperref}
\hypersetup{colorlinks=true}
\usepackage{amsmath}
\usepackage{amsthm}
\usepackage{biblatex}
\addbibresource{bibliography.bib}

\setbeamertemplate{caption}[numbered]

\newtheorem{proposition}{Proposition}
\def\P{\mathbb P}
\def\ind{\perp\!\!\!\perp}
\DeclareMathOperator*{\argmax}{arg\,max}

\title[Forecasting the Solar X-Ray Flux]{Forecasting the Solar X-Ray Flux}

\author{Victor Verma, Yang Chen, Stilian Stoev}
\institute[]
{
Department of Statistics \\
University of Michigan
}
\date[6/22/23]{6/22/23}

\begin{document}

\begin{frame}
    \titlepage
\end{frame}

%\begin{frame}{Outline}
%    \tableofcontents
%\end{frame}

\begin{frame}{Background on Solar Flares}
    \begin{itemize}
        \item A solar flare is a sudden, massive eruption of electromagnetic radiation from the Sun's atmosphere.
        \item Adverse effects of solar flares: 
        \begin{itemize}
            \item Radio blackouts
            \item Coronal mass ejection (CME) $\rightarrow$ electromagnetic pulse $\rightarrow$ electrical blackouts
            \item Solar energetic particle event (SEP) $\rightarrow$ irradiation of astronauts
        \end{itemize}
        \item Some notable incidents:
        \begin{itemize}
            \item 1859: The Carrington flare, one of the most extreme ever recorded
            \item 1989: The electrical grid in Quebec was shut down for several hours
            \item 2022: Dozens of Starlink satellites were destroyed
        \end{itemize}
    \end{itemize}
\end{frame}

\begin{frame}{Flare Strength}
    \begin{itemize}
        \item Flare classes: A, B, C, M, and X
        \item Classification is based on peak soft X-ray flux
    \end{itemize}
    \begin{table}
        \centering
        \begin{tabular}{|c|c|}
            \hline
            Flare Class & X-Ray Flux Threshold ($\text{W} / \text{m}^2$) \\
            \hline
            A & $10^{-8}$ \\
            B & $10^{-7}$ \\
            C & $10^{-6}$ \\
            M & $10^{-5}$ \\
            X & $10^{-4}$ \\
            \hline
        \end{tabular}
        \caption{X-ray flux thresholds for the different flare classes}
        \label{tab:flare_classes}
    \end{table}    
\end{frame}

\begin{frame}{Goal}
    \textbf{Goal: predict whether the X-ray flux will be above a given threshold}
\end{frame}

\begin{frame}{X-Ray Flux Data}
    \begin{figure}
        \centering
        \includegraphics[scale=0.5]{flare_flux_example.png}
        \caption{The X-ray flux for an X-class flare}
        \label{fig:flare_flux_example}
    \end{figure}
\end{frame}

\begin{frame}{X-Ray Flux Data}
    \begin{itemize}
        \item The flux level is measured by the Geostationary Orbiting Environmental Satellites (GOES).
        \item We use data from the GOES-15 satellite, which has data from 3/31/10-3/4/20.
        \item The flux is measured every 2s; we use 1m averages.
    \end{itemize}
\end{frame}

\begin{frame}{X-Ray Flux Data}
    \begin{figure}
        \centering
        \includegraphics[scale=0.5]{flux_time_series.png}
        \caption{The GOES-15 X-Ray flux time series}
        \label{fig:flux_time_series}
    \end{figure}
\end{frame}

\begin{frame}{X-Ray Flux Data}
    \begin{figure}
        \centering
        \includegraphics[scale=0.5]{flux_20170906.png}
        \caption{The X-Ray flux, 9/3/17-9/9/17.}
        \label{fig:flux_20170906}
    \end{figure}
\end{frame}

\begin{frame}{X-Ray Flux Data}
    % latex table generated in R 4.2.3 by xtable 1.8-4 package
    % Thu Jun 22 04:54:14 2023
    \begin{table}[ht]
    \centering
    \begin{tabular}{cccc}
      \hline
    Good Data? & Is Not Missing? & Is Above Min? & Proportion \\ 
      \hline
    Yes & Yes & Yes & 0.74 \\ 
      Yes & Yes & No & 0.17 \\ 
       & No &  & 0.055 \\ 
      No & No &  & 0.024 \\ 
      No & Yes & No & 0.012 \\ 
      No & Yes & Yes & 0.0041 \\ 
      Yes & No &  & 3.8e-07 \\ 
       \hline
    \end{tabular}
    \end{table}
\end{frame}

\begin{frame}{X-Ray Flux Data}
    \begin{figure}
        \centering
        \includegraphics[scale=0.5]{hill_plot.png}
        \caption{A Hill plot for the X-Ray flux.}
        \label{fig:hill_plot}
    \end{figure}
\end{frame}

\begin{frame}{SHARP Parameter Data}
    Solar flares typically emerge from active regions, areas of the solar atmosphere  characterized by intense magnetic activity.
    \begin{figure}
        \centering
        \includegraphics[scale=0.33]{active_region.png}
        \caption{Source: \cite{toriumi2019flar}}
        \label{fig:active_region}
    \end{figure}
\end{frame} 

\begin{frame}{SHARP Parameter Data}
    \begin{itemize}
        \item SHARP (Spaceweather HMI Active Region Patch) parameters are variables that describe active regions.
        \item An HMI Active Region Patch (HARP) is an automatically-delineated area that encloses an active region.
        \item We use SHARP data from 5/1/10-6/24/18.
        \item SHARP parameters are calculated every 12m.
    \end{itemize}
\end{frame}

\begin{frame}{SHARP Parameter Data}
    \begin{figure}
        \centering
        \includegraphics[scale=0.19]{magnetogram_20170906.png}
        \caption{HARPs on 9/6/17.}
        \label{fig:magnetogram_20170906}
    \end{figure}
\end{frame}

\begin{frame}{SHARP Parameter Data}
    \begin{figure}
        \centering
        \includegraphics[scale=0.55]{sharp_params_20170906.png}
        \caption{SHARP parameters, 9/3/17-9/9/17}
        \label{fig:sharp_params_20170906}
    \end{figure}
\end{frame}

\begin{frame}{Preprocessing}
    \begin{itemize}
        \item Different ways to put the SHARP parameters on the same scale:
        \begin{itemize}
            \item Standardization
            \item Rank-transforming
            \item Pareto-transforming
        \end{itemize}
        \item Covariates are lagged:
        \begin{figure}
            \centering
            \includegraphics[scale=0.4]{unlagged_vars.png}
            \includegraphics[scale=0.4]{lagged_vars.png}
        \end{figure}
        \item Dimensionality reduction
        \begin{itemize}
            \item PCA
            \item Spline-based dimensionality reduction 
        \end{itemize}
    \end{itemize}
\end{frame}

\begin{frame}{Preprocessing}
    \begin{figure}
        \centering
        \includegraphics[scale=0.5]{spline_plots.png}
        \caption{A sequence of flux values and its B-spline basis approximations.}
        \label{fig:spline_plots}
    \end{figure}
\end{frame}

\begin{frame}{Optimal Prediction}
    \begin{itemize}
    \item General problem: given a response $Y$ and covariates $X$, predict whether $Y > F_Y^{-1}(p)$ using $X$.
    \item Idea: predict that $Y > F_Y^{-1}(p)$ when $g(X) \in B$ for some $g$, $B$.
    \item $g(X)$ should be calibrated, i.e.,
    \[
    \P(g(X) \in B) = \P(Y > F_Y^{-1}(p)) = 1 - p.
    \]
    \item Set $B = [F_{g(X)}^{-1}(p), \infty)$, i.e., predict that $Y > F_Y^{-1}(p)$ when $g(X) > F_{g(X)}^{-1}(p)$.
    \item \textbf{What should $g$ be?}
    \end{itemize}
\end{frame}

\begin{frame}{Optimal Prediction}
    \begin{theorem} The optimal predictor of $1_{\{Y > F_Y^{-1}(p)\}}$ in terms of $X$ is of the form $1_{\{h(X) > F_{h(X)}^{-1}(p)\}}$, where
    \[
    h(x) = \frac{f_1(x)}{f_0(x)} = 
    \frac{p \int_{F_Y^{-1}(p)}^{\infty} f_{X, Y}(x, y)\,dy}{(1 - p)\int_{-\infty}^{F_Y^{-1}(p)} f_{X, Y}(x, y)\,dy}.
    \]
    More precisely, for every other calibrated predictor $1_{\{g(X) > F_{g(X)}^{-1}(p)\}}$, we have
    \begin{align*}
    \P[Y > F_Y^{-1}(p) \mid g(X) > F_{g(X)}^{-1}(p)] \le \P[Y > F_Y^{-1}(p) \mid h(X) > F_{h(X)}^{-1}(p)]
    \end{align*}
    \end{theorem}
\end{frame}

\begin{frame}{Optimal Prediction}
    \begin{corollary} Consider the additive error model
    \[
    Y = m(X) + \sigma(X)\epsilon,
    \]
    where $X \ind \epsilon$ and $\sigma(X) > 0$.
    The optimal predictor is 
    \[
    h(x) = \frac{m(x) - F_Y^{-1}(p)}{\sigma(x)}.
    \]
    \end{corollary}
\end{frame}

\section{Prediction Methodologies}

\begin{frame}{Tail Dependence Approach}
    The \textbf{tail dependence coefficient} between random variables $U$ and $V$ is defined as
    \[
    \lambda(U, V) = \lim_{p \to 1^{-}} \P[V > F_V^{-1}(p) \mid U > F_U^{-1}(p)].
    \]
    Note that
    \begin{align*}
        \lambda(U, V) &= \lim_{p \to 1^{-}} \frac{\P[U > F_U^{-1}(p), V > F_V^{-1}(p)]}{\P[U > F_U^{-1}(p)]} \\
        &= \lim_{p \to 1^{-}} \frac{\P[U > F_U^{-1}(p), V > F_V^{-1}(p)]}{1 - p} \\ 
    \end{align*}
\end{frame}

\begin{frame}{Tail Dependence Approach}
    Idea: find $g$ so that $g(X)$ and $Y$ have high tail dependence, predict using $g(X)$.

    \medskip
    
    One estimator of the tail dependence coefficient:
    \[
    \hat{\lambda}_p(Y, g(X)) = \frac{1}{n(1 - p)}\sum_{i = 1}^n I(g(X_i) \ge \hat{F}_{g(X)}^{-1}(p))I(Y_i \ge \hat{F}_{Y}^{-1}(p));
    \]
    $p$ is fixed, and $\hat{F}_{g(X)}^{-1}(p)$ and $\hat{F}_{Y}^{-1}(p)$ estimate the $p$th quantiles of $g(X)$ and $Y$, respectively.
\end{frame}

\begin{frame}{Tail Dependence Approach}
    Given data $(X_1, Y_1), \ldots, (X_n, Y_n)$ and some class $\mathcal{C}$ of functions, compute $g$ as 
    \[
    \argmax_{\tilde{g} \in \mathcal{C}} [\hat{\lambda}_p(Y, \tilde{g}(X)) + \text{penalty}].
    \]
    Some choices for $\mathcal{C}$:
    \begin{itemize}
        \item $\mathcal{C}_{\text{LM}} = \{\tilde{g} : \tilde{g}(x) = x^{\top}\beta, \beta \in \mathbb{R}^d\}$
        \item $\mathcal{C}_{\text{GAM}} = \left\{\tilde{g} : \tilde{g}(x) = \sum_{i = 1}^d s_i(x_i)\right\},$
        where the $s_i$'s are $k$-knot natural cubic splines.
    \end{itemize}
    If $\mathcal{C}$ is $\mathcal{C}_{\text{LM}}$ or $\mathcal{C}_{\text{GAM}}$, then
    \[
    \text{penalty} = \eta\|\beta - \hat{\beta}_{\text{prev}}\|_2.
    \]
\end{frame}

\begin{frame}{Tail Dependence Approach}
    Prediction works as follows. With the data $(X_1, Y_1), \ldots, (X_n, Y_n)$,
    \begin{itemize}
        \item set $\hat{F}_{g(X)}^{-1}(p)$ to the $p$th sample quantile of $\{g(X_1), \ldots, g(X_n)\}$
        \item given a new observation $X_*$, predict that $Y_* > F_{Y}^{-1}(p)$ if $g(X_*) > \hat{F}_{g(X)}^{-1}(p)$
    \end{itemize}
\end{frame}

\begin{frame}{TSS}
    The True Skill Statistic, used to measure performance, is
    \[
    TSS = \frac{TP}{TP + FN} - \frac{FP}{FP + TN}
    \]
\end{frame}

\begin{frame}{Tail Dependence Approach}
    \begin{figure}
        \centering
        \includegraphics[scale=0.5]{group02_cv_study01_tss.png}
        \caption{Cross-validation results, $\mathcal{C}_{\text{LM}}$.}
        \label{fig:group05_cv_study01_tss}
    \end{figure}
\end{frame}

\begin{frame}{Tail Dependence Approach}
    \begin{figure}
        \centering
        \includegraphics[scale=0.5]{group05_cv_study01_tss.png}
        \caption{Cross-validation results, $\mathcal{C}_{\text{LM}}$, 15 B-splines.}
        \label{fig:group05_cv_study01_tss}
    \end{figure}
\end{frame}

\begin{frame}{Future Work}
    \begin{itemize}
        \item Tail dependence approach: consider functions $g(X)$ that are copulas or neural networks
        \item Logistic regression
        \item State space models
    \end{itemize}
\end{frame}

\begin{frame}{References}
    \nocite{*}
    \printbibliography
\end{frame}

\end{document}